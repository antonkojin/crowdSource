\documentclass[a4paper]{article}

%% Language and font encodings
\usepackage[italian]{babel}
\usepackage[utf8x]{inputenc}
\usepackage[T1]{fontenc}

%% Sets page size and margins
\usepackage[a4paper,top=3cm,bottom=2cm,left=3cm,right=3cm,marginparwidth=1.75cm]{geometry}

%% Useful packages
\usepackage{amsmath}
\usepackage{graphicx}
\usepackage[colorinlistoftodos]{todonotes}
\usepackage[colorlinks=true, allcolors=blue]{hyperref}
\usepackage{float}
\usepackage[normalem]{ulem}

\title{Manuale progetto database}
\author{Anton Kozhin}

\begin{document}
\maketitle

\section{Dipendenze}
Installare i seguenti software seguendo le relative guide
\begin{itemize}
\item Docker: \verb|https://www.docker.com/community-edition|
\item Docker compose: \verb|https://docs.docker.com/compose/install|
\end{itemize}

\section{Struttura cartelle}
\renewcommand{\labelitemi}{$-$}
\renewcommand{\labelitemii}{$-$}
\renewcommand{\labelitemiii}{$-$}
\renewcommand{\labelitemiv}{$-$}
\begin{itemize}
  \item docker-compose.yml: startup directives
  \item db
  \begin{itemize}  
    \item sql: schema, funzioni e dati
    \item Dockerfile  
  \end{itemize}
  \item web
  \begin{itemize}
    \item Dockerfile
    \item webapp
    \begin{itemize}
      \item routes: endpoints
      \item lib/db.js: interfaccia database
      \item views: mustache views
      \item scss: scss files
      \item keys: SSL keys
      \item public
      \begin{itemize}
        \item js: client side javascript
        \item css: librerie esterne
        \item img: risorse grafiche
      \end{itemize}
    \end{itemize}
  \end{itemize}
\end{itemize}

\section{Configurazione e avvio}
Dopo esservi procurato un certificato SSL/TLS, copiatelo nella cartella \verb|web/webapp/keys|. \`E necessario per una comunicazione sicura attraverso HTTPS.\\\\
Per avviare l'applicazione eseguire il comando \verb|docker-compose up|'.

\section{Utilizzo}
\subsection{Account amministratore}
Per confermare la registrazione dei requester, \`e necessario collegarsi alla pagina web \verb|admin/| ed eseguire il login con la password predefinita  ``$0$''. Successivamente al login sar\`a possibile cambiare la password dell'amministratore.

\subsection{Worker e Requester}
Gli utenti dell'applicazione dovranno registrarsi sulla pagine \verb|signup/|.
successivamente potranno fare il login sulla pagina \verb|login/| che li reindirizzer\`a alle relative dashboard.

\subsubsection{Requester}
I requester, dopo la registrazione, dovranno attendere la verifica da parte dell'amministratore per poter utilizzare l'applicazione. Prima che questo avvenga essi verranno bloccati al login con un avviso.

La pagina principale del requester gli permette di creare nuove campagne di lavoro e di reperire statistiche sulle campagne esistenti.

\subsubsection{Worker}
La pagina principale del worker gli permette di entrare a far parte di nuove campagne di lavoro e di richiedere un task da eseguire per le campagne disponibili.

\end{document}
